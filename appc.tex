\chapter{Analysis on the choice of $f(x)$}

The function $f(x_{ij})$ includes Error function with a linear argument. Error function is defined as

\begin{equation}
erf(x) = \frac{2}{\sqrt{\pi}} \int_0^x e^{-t^{2}} dt.
\label{erfDefinition}
\end{equation}

It is zero at x = 0 and goes asymptotically towards -1 and 1. Appropriate choice for
$f(x_{ij})$ can be obtained by shifting and scaling of standard erf(x). Say, we take
a function of the following form
\begin{equation}
f(x) = a(erf(b(x-c)) + 1)
\label{fx_function}
\end{equation}
where a, b, c are real numbers. Here, f(x) is zero at x = c and it goes from 0 to 2a. To
control the slope of $f(x)$ during the transition, let us calculate $f'(x)$ as

\begin{eqnarray}
\frac{d}{dx}f(x) & = & \frac{d}{dx}a(erf(b(x-c)) + 1)\\
&  = & \frac{d}{dx}a(\frac{2}{\sqrt{\pi}} \int_0^{b(x-c)} e^{-t^{2}} dt + 1)\\
&  = & \frac{2a}{\sqrt{\pi}} \frac{d}{dx}(\int_0^{b(x-c)} e^{-t^{2}} dt)\\
&  = & \frac{2a}{\sqrt{\pi}} e^{-(b(x-c))^{2}} \frac{d}{dx}({b(x-c)})\\
&  = & \frac{2ab}{\sqrt{\pi}} e^{-(b(x-c))^{2}}
\end{eqnarray}

As $\frac{d}{dx}f(x)|_{x = c}$ = $\frac{2ab}{\sqrt{\pi}}$, for a constant shift of 2a,
increasing b would increase the slope of the sigmoid function $f(x)$. In fact the sigmoid
function will tend to a step function as b \to \infty.

Truncation error in interpolation is due to the non-zero value of $f(x)$ at the
boundary, t \in $\{t_{ij}^{0},t_{ij}^{1}\}$ mentioned in Eq. \ref{paramEquation}.
Increasing the slope of $f(x)$ at $x=c$ will lead to a quicker convergence of the value
of $f(x)$ at the boundary and thus reducing the truncation error.

%Another method where instead of solving an eigenvalue problem as in CTM,
If the diabatic curves are shifted to match at the seam then it would be
a special case of CTM with a big enough value of b so that erf(x) becomes a step
function. Hence, shifting at the seam is just a lower level of the same
method where the interpolated curve is not smoothened. After shifting the diabatic
states to match each other at the seam, the one with the lowest potential
energy among all is chosen on either sides of the seam.
\newpage
