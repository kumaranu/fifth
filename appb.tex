\chapter{Further computational efficiency for reduced-dimensional potential
surface calculations by examining the span of the invariant subspace of the
the topological overlap matrix}

As mentioned earlier that fragmentation based electronic structure has made it possible to get
post Hartree-fock accuracy on relatively large systems with an computational expense of DFT methods.
As potential energy surface calculations involve these electronic structure calculations at each
point on a grid, the computational expense grows exponentially. Use of parallel computing in the
fragment based electronic structure calculations has made it possible to reduce the
computational time at each point on the grid. However there is still a lot of scope to improve the algorithms to
get the same accuracy at a lower computational expense. One of these options reusuability has been utilized here
and its implementation is explained below.

Say, there exists a molecular system with P number of primary and D number of derivative fragments.
Let's only consider a 1D potential energy
surface with M grid points. Let the average cpu time for the electronic structure calculations of primary
fragments at lower and higher level are $t_{primary}^{low}$ and $t_{primary}^{high}$ respectively. Similarly,
the average cpu times for the electronic structure calculations of derivative fragments at lower level,
derivative fragments at higher level and full system at lower level are  $t_{derivative}^{low}$ ,
$t_{derivative}^{high}$ and $t_{Full}^{low}$ respectively.

The total CPU time for this potential energy scan (say, $T_{1}$) with $M$ grid points
would be equal to:
\begin{equation}\label{eq:withoutReusabilityEqn1}
\left.\begin{aligned}
T_{1} &     =  ($t_{Full}^{low}$ + $P.t_{primary}^{high}$ + $D.t_{derivative}^{high}$ + $P.t_{primary}^{low}$
 + D.t_{derivative}^{low}) M
\end{aligned}\right.
\end{equation}

Say, we know that only p out of P primary fragments and d out of D derivative fragments are changing
during a potential energy scan, so we can avoid the repeated calculations. We can do all the
 electronic structure calculations for the first step on the grid and avoid
the calculations of non-changing fragments for the other M-1 points on the grid. This way, the total CPU time for
potential energy scan (say, $T_{2}$) would be equal to:
\begin{equation}\label{eq:withReusabilityEqn2}
\left.\begin{aligned}
T_{2} &     =  P.t_{primary}^{high} + D.t_{derivative}^{high} + P.t_{primary}^{low} 
      + D.t_{derivative}^{low} + (t_{Full}^{low} + p.t_{primary}^{high}\\
      &  ~~~ + d.t_{derivative}^{high} + p.t_{primary}^{low} + d.t_{derivative}^{low}) (M-1)
\end{aligned}\right.
\end{equation}

The difference between $T_{1}$ and $T_{2}$ is equal to:
\begin{equation}\label{eq2}
\left.\begin{aligned}
T_{1} - T_{2} &     =  ((P-p)(t_{primary}^{high} + t_{primary}^{low}) + (D-d)(t_{derivative}^{high} +
 t_{derivative}^{low}))(M-1)
\end{aligned}\right.
\end{equation}

Considering the example of a 1D potential energy scan with 200 grid points (M = 200) for 12-water water-wire ($(H_{2}O)_{12}H^{+}$),
if primary fragments are chosen to be the water dimers formed by the neighbouring water molecules and their overlaps as
derivative fragments. There are 11 primary fragments (P = 11) and 10 derivative fragments (D = 10). During the potential energy scan
2 of the primary (p = 2) and 1 of the derivative (d = 1) fragments are changing.
Therefore the CPU time saved is equal to:
\begin{equation}\label{eq:withReusabilityEqn3}
\left.\begin{aligned}
CPU time saved & =  9((t_{primary}^{high} + t_{primary}^{low}) + (t_{derivative}^{high} + t_{derivative}^{low}))\times199
\end{aligned}\right.
\end{equation}

This is a huge saving of computational time for the 1D PES case. This option for reusability is even more helpful in calculating
the higher dimensional PES as the number of fragments changing in those cases are also small as compared to the total number of
fragments. Similar to the development of Message Passing Interface (MPI) based fragment electronic structure methods
\cite{fragAIMD-elbo,fragAIMD-CC}, a smart parallelization based program can be utilized. One of the good
implementation for this reusability program would be to write a program which runs the full system
calculations in parallel and computes the PES after all the calculations are finished.

\newpage
