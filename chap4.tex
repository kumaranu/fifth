\chapter{Conclusion and future work}

\section{Using fragmentation with sampling}
{\label {samplingSection}}
Although AIMD has shown to be excellent for small to medium sized molecular
systems, but it is still impossible to use it for large systems such as biological
molecules, ion channels with thousands of atoms etc. One of the reasons for this limitation
is the electronic structure calculations and another reason is that the protein
structural changes occur in the time scales of micro- to milliseconds. Getting a micro
or millisecond trajectory using AIMD is inaccessible even for smaller molecules. Though,
there exist several MD techniques to sample these rare events. But MD lacks quantum
nuclear effects which are crucial for hydrogen transfer reactions in biological systems.
Catalytic oxidation of linoleic acid by an enzyme, Soybean lipoxynase-1 (SLO-1) is one
of these examples \cite{SLO-rare-isotopes, SLO1-Rareevents}. H/D nuclear tunneling has
been observed in this oxidation reaction by studying kinetic isotope effects. It has
been shown in Refs. \onlinecite{SLO-rare-isotopes} that the acceptor side in the above
mentioned oxidation reaction is an octahedral $Fe^{3+}$ complex with OH and isoleucine
as two of the six ligands. During the proton transfer from carbon of linoleic acid to
oxygen of the hydroxyl ligand of the acceptor, hydrogen of hydroxyl group makes a
hydrogen bond with oxygen of carbonyl group attached to isoleucine. Hence there is a
need to study the correlated potential energy surfaces as a function of reaction
coordinate. The correlated potential should include both the protons, proton getting
transferred and the proton of hydroxyl group simultaneously.
Potential energy surface needs to be approximated by QM/MM calculations on a grid.
As the number of calculations required to calculate potential energy on the whole
grid grow exponentially with the number of degrees of freedom, sampling functions
are used to sample important points
of the potential so that a small number of calculations can be performed to get an
interpolated potential with some acceptable accuracy. These sampling functions use
quantum wavepacket density, approximate potential, gradients and local entropy
of quantum wavepacket calculated using Shannon information theory \cite{shannonIsoprene}.
After sampling
the important points from the grid, there remains another problem and i.e. calculating
the QM part of the hybrid QM/MM calculations. QM can be further divided into
fragmentation based electronic structure calculations and post-hartree fock
accuracy can be achieved for the important regions of the proton transfer molecular
system. As there are proton transfer reactions, hydrogen atom will change fragmentation
schemes and hence multiple topologies might be used in these fragment based electronic
structure calculations. Different type of interpolation schemes such as local Hermite
curve interpolation, low-pass filtered Lagrange interpolation and monomial symmetrization
approximation might be used to get interpolated potential surfaces for each of the
topologies. CTM interpolation can then be used on the topology potentials to get a
multi-topology based potential energy surface.

\section{Landau-Zener based topological transition probabilities} 
Under the assumption that the wavefunctions corresponding to the two diabatic topological states
are orthogonal, at a given time step the probability of hopping from one topological state to another can be written
as

\begin{equation}
\label{LandauZEquation}
P = \exp\bigg[-\frac{2\pi\epsilon_{12}^{2}}{\hbar\nu|F_{1}-F_{2}|}\bigg]
\end{equation}

where $\epsilon_{12}$ is the off-diagonal element of the Hamiltonian matrix discussed above at the time step, $\nu$
is the velocity of the whole system calculated at the time step and $F_{1}$ and $F_{2}$ are the forces calculated
using topologies one and two at the time step respectively.

If the off-diagonal element is a function of the difference in energy of the two states such that 
the $\epsilon_{12}$ is bounded under $|\epsilon_{1} - \epsilon_{2}| \rightarrow 0$ then as the two
states come closer and $\nu \rightarrow \infty$, P $\rightarrow$ 1. And also as $\nu \rightarrow 0$
and if the off-diagonal element is finite under $|\epsilon_{1}-\epsilon_{2}|$ 
then P $\rightarrow$ 0.

Eq. \ref{LandauZEquation} can be transformed into the following equation:
\begin{equation}
P = \exp\bigg[-\frac{2\pi\epsilon_{12}^{2}}{\hbar|\frac{dV_{1}}{dt}-\frac{dV_{2}}{dt}|}\bigg]
\end{equation}

as $\nu$ is the velocity of the system and $F_{i} = \frac{dV_i}{dR}$ and hence $\nu*F_{i}$ is $\frac{dV}{dt}$.

\section{Stochastic surface hopping based multi-topology scheme}
One of the key limitations of classical trajectory method is that it can only be used for
systems for which the nuclear motion can be represented by only one potential energy
surface. In case of fragment based AIMD this potential is approximated by employing one
fragmentation topology with the use of Eq. \ref{eq:pie-oniom} and Eq. \ref{oniombasic-2}.
Molecular collision processes involving nonadiabatic transitions such as charge transfer,
quenching, radiationless transitions, etc. cannot be represented well by using classical
trajectory method. Avoided crossing of adiabatic states causes non-adiabatic effects
\cite{landau1932theory, zener1932non, stueckelberg1932ecg} during dynamics and surface
hopping methods are known to work well in obtaining effective potential.
Similarly, when a proton hops between one topology surface to another an interpolation is
required to get effective potential surface which can be used during dynamics. Analogous to
topology change in fragment based AIMD or potential surface discussed in chapter 2, the
non-adiabatic coupling occur in small regions of configurational space. 
Under the semi-classical treatment of non-adiabatic transitions, the probability $|a_{j}|^2$
that a system is in state $j$ at time $t$ can be calculated by solving the differential equation
\begin{equation}
\label{surfaceHoppingEqn}
{\dot a}_{j} = - \sum_{i}^{}a_{i}v \big \langle \phi_{j}|\frac{\partial \phi_{i}}{\partial R}
 \big \rangle exp\bigg[-(i/\hbar)\int_{}^{t} (E_{i}-E_{j})dt  \bigg]
\end{equation}
for a given trajectory R(t) \cite{tully1971trajectory}. Here $v$ is nuclear velocity,
$\phi_{i}$ are electronic eigenstates. $E_{i}$ and $E_{j}$ are electronic energy eigenvalues
corresponding to $i^{th}$ and $j^{th}$ states respectively. Hence, following our analogy
between electronic energies and fragment based electronic energies, the surface hopping method
can utilized with an approximation. The approximation is the assumption that the eigenstates
corresponding to different topologies are orthogonal and then, Eq. \ref{surfaceHoppingEqn}
can be solved to get $a_{j}$, who square would give the probability corresponding to the
$j^{th}$ topology.

%\section{Multidimensional seams and their complexity with the number of protons involved in the
%scan}
%To solve the general problem in fragment based potential surface calculations, we have to interpolate
%between M number of N-dimensional potential surfaces with any number of hops distributed over
%the whole domain of the potential surface. So far we have done a lot of one dimensional potential
%interpolations on different waterwires such as $(H_2O)_{12}H^+$, $(H_2O)_{5}H^+$ and $(H_2O)_{3}H^+$
%along with other water clusters such as eigen cations. All of these one dimensional interpolations are
%shown to work well in Section \ref{eigenResults} and Section \ref{waterwireResults} of Chapter
%\ref{chapter3}. Hence we comprehend that the errors in potentials should be of the order of a tenth of
%a kcal/mol with known hopping points. But in the multidimensional case, we don't have a prior knowledge of the
%seam and hence we need to solve the problems with unknown seams.
%\subsection{Dimensionality of seams}
%For one-dimensional potentials the seam is a point for every hop between two topologies but it more
%complex for higher dimensional potential surface hopping. For example, in two dimensional potentials
%the seam is a curved line which represents the intersection of two potential surfaces. Hence for an
%N-dimensional potential
%surface, the seam would be an N-1 dimensional surface and we would not know its prior location in the
%N-dimension space. One of the future goal of this work is to utilize sampling methods as mentioned
%earlier in Section \ref{samplingSection}, to come up with fewer number of points to do electronic
%structure calculation to generate the desired potential surface. Sampling functions give irregularly
%arranged set of points on the grid. We also need to make sure that the interpolation works well with
%such as fewer number of grid points along the with the unknown seam. One of the possible approaches
%could be coming up with a modification in the Shannon entropy function to identify seams.
%
%\subsection{Estimate for seams in multi-dimensions}
%Coupled proton motions involve more than two topologies in their multi-topology interpolation scheme.
%As illustrated by the two dimensional Morse potential problem solved in Section \ref{analyticalPotentials}
%of Chapter \ref{chapter3} where we have 5 seams, this is complex problem with unknown seams. Hence, we need
%to come up with estimates about the seam for every hop between two different potential surfaces. One of the
%methods in this regard can be using Delaunay triangulation of the whole space by using the ``centroids'' of
%each topology as the nodes. The boundaries (which may not be in general be lines as in higher than two
%dimensional potentials) would correspond to the direction of hop and a region near center of each edge can
%be considered as seam.










%\newpage
