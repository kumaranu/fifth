\chapter{Introduction}

In molecular dynamics the nuclear framework is propagated in time using forces obtained
from potential energy gradients. For large systems in equilibrium, this may be done
using empirical potential energy functions, with parameters determined from experiment
\cite{Allen, Ciccotti, Haberlandt, Frenkel, Binder, Berne, Esser}. For reactive systems
however, $\it{ab~ initio}$ strategies may be essential. For the study of an excess proton
in water \cite{James, VothDay, IyengarPetersen, fragAIMD-elbo, fragAIMD-CC,fragAIMD},
however, methods such as \cite{polarizability1, polarizability2, aqc} EVB
(empirical valence bond) and MSEVB (Multi-state EVB) \cite{VothDay} allow special
handling; but these are not easily generalized for general reactive systems and
$\it{ab~ initio}$ molecular dynamics (AIMD) methods present a viable, but expensive
computational option. In AIMD, electronic structure calculations are performed to obtain
potential energies and gradients in order to propagate the nuclei in time.

Due to the intrinsic scaling of the electronic structure methods, such as 
$O(N^{3.5})$ for DFT, $O(N^5)$ for MP2, and $O(N^7)$ for CCSD(T), AIMD calculations
have thus far been restricted to the use of density functional treatment 
\cite{Jutter-2D-MOFs-on-water-DFT-AIMD,Gerber-AIMD-atmospheric-2,Gerber-AIMD-atmospheric,
Hutter-Review,Kirchner-Hutter-Aimd-Apps}. As the size of system grows the computational
expense becomes a bottleneck for the application of post Hartree-fock methods in AIMD.

There are essentially two flavors of ``on-the-fly'' dynamics where the gradients from
electronic structure are employed to propagate the nuclear framework. These include:
(a) Born-Oppenheimer molecular dynamics (BOMD) \cite{KarplusBO1, Leforestier-BO,
Helgaker-AIBO, BO-Hase, Hessian, bernyPESreview, Hrant-Berny-PESreview, formaldehyde,
pulay, CPchargedsystems} and (b) extended Lagrangian methods \cite{AndersenEL,
ParrinelloEL, Parrinello1984jcp-PI}, of which the Car-Parrinello method
\cite{CP, Madden, Marx-Cpreview, Cpdensity, Cptuckerman1, Cptuckerman3, Cpcpks, Carpar89,
Galli91, Parrinello-harmonic, Na+solv, NH4+solv, K+solv, SO3solv, Be2+solv, H+OH-solv,
Parrinello:ClH2O6, CarPar-Laasonen, CP-Wannier1, Marzari-Vanderbilt-Wannier, Kohn-Wannier,
CP-Gaussians1, CP-Gaussians2, CP-Gaussians-Martyna, Galli-mu-problem-BOCP, Galli-mu-problem,
carter1, carter2, carter3,Parrinello1984jcp-PI} is perhaps the
most commonly used. The BOMD method utilizes electronic Schr\"{o}dinger equation to
compute nuclear gradients that are employed at each step during molecular dynamics
and simulate classical dynamics on the Born-Oppenheimer surface. The extended Lagrangian
methods, on the contrary, simultaneously propagate the electronic structural
parameters, along with the nuclear framework, through a simple adjustment of
time-scales. The Car-Parrinello method is one of the most commonly employed approach
of this kind, where plane-wave basis functions are used to describe the valence
electrons and pseudopotentials describe core electrons. By contrast, the atom-centered
density matrix propagation (ADMP) \cite{admp1,admp2,admp3,admp4,IyengarFrisch,
IyengarSchlegelVoth} method uses atom-centered Gaussian basis functions and single
particle electronic density matrices as electronic parameters that are propagated
with the nuclei. The ADMP method has been employed to study vibrational spectroscopy
in hydrogen bonded systems beyond the Harmonic approximation
\cite{IyengarProtonatedWater1,IyengarProtonatedWater2,DietrickIyengar,LiTeigeIyengar,
LiOomensEyler,IyengarLiSumner}, proton conduction through biological ion-channel
studies \cite{RegaIyengar}, hydrogen transfer reaction in biological
\cite{SLO1-Rareevents} and atmospheric \cite{VimalPacheco,PachecoDietrick,
DietrickPachecoPhatak} problems. Importantly, ADMP was used to discover the
amphiphilic nature of the hydrated proton \cite{PetersenIyengar,IyengarDayVoth}
and was later confirmed by several experiments \cite{PetersenIyengar,IyengarDayVoth,
PetersenSaykally1,PetersenSaykally2,LeveringSierra,MuchaFrigato,KudinCar}. ADMP is
computationally cheaper than BOMD as it only requires one SCF (self-consistent field)
cycle instead of complete SCF convergence ($\approx$ 8-12 SCF steps depending on
complexity of systems, when using incremental Fock-matrix formation) for electronic
structure calculations. Although the computational expense is divided into many other
calculations such as gradient calculations which are common to both ADMP and BOMD. BOMD
is converged at each time step during dynamics, hence larger than ADMP time steps can
be used in BOMD. In practice, it turns out that ADMP is twice as fast as BOMD for the
cases where convergence is not a problem for BOMD \cite{BrorsenZahariev}.

In a recent set of publications \cite{fragAIMD,fragAIMD-elbo}, it
has been shown how $\it{ab~ initio}$ molecular dynamics simulations can be performed
at accuracies comparable to CCSD and MP2 levels with DFT cost. This has been
shown for both extended Lagrangian and
Born-Oppenheimer versions of AIMD. The approach has been to employ fragment
based electronic
structure \cite{morokuma-review1,Raghavachari-review,morokuma-review2,collins-review,
gordon-review,Weitao-DAC,oniom,oniomxs,EFP,FMO-feature-article,mfcc,
ee-mbe-water,ee-mbe,gebf,LS3DF,mta,ONIOM-XO,mim,mob,herbert-gmbe-1,
herbert-gmbe,Hirata-Fragments,Hirata-Fragments-2,gordon-fmo-dynamics2,
gordon-dynamics-1,cfm,Gao-XPol}. Specifically, in
Refs. \cite{fragAIMD,fragAIMD-elbo,fragAIMD-CC}, the well-known ONIOM
method \cite{oniom} has been generalized by using the set-theoretic,
inclusion-exclusion principle \cite{pie} and allows multiple, overlapping
``model'' systems that may cover the entire domain of the full
system. Our approach is closely related to those from
Refs. \cite{mim,mta,ONIOM-XO} and in
Refs. \cite{fragAIMD,fragAIMD-elbo}, we have used this approach to
compute Born-Oppenheimer dynamics trajectories that agree with MP2-based
AIMD. In Refs. \cite{fragAIMD-elbo,fragAIMD-CC}, extended-Lagrangian versions 
of the same have been introduced and these are in agreement with MP2 and
CCSD based Born-Oppenheimer molecular dynamics. In addition, in
Ref. \cite{fragAIMD,fragAIMD-elbo}, it has been shown that potential
surfaces in agreement with CCSD(T) can be obtained for protonated water
clusters with computational effort that scales as DFT. The Born-Oppenheimer
version of this approach is called frag-BOMD, whereas, the extended-Lagrangian
generalization is called atom-centered density matrix propagation with
post-Hartree-Fock accuracy (ADMP-pHF) since it is derived from the
ADMP \cite{admp1,admp2,admp3,admp4} Lagrangian.

In addition to classical trajectories, in many problems of interest in
chemical physics, it is often also required to compute molecular
potential energy surfaces. The application of this fundamental concept in
theoretical chemistry is quite broad and applies to a range of topics
including, the study of nuclear quantization in chemical reactions
\cite{polarizability1,dikenHeadrick}, reaction path sampling
\cite{Bolhuis,Dellago,Elber,Gillilan,Henkelman,MacFadyen,Voter,Carter,
VoterChemPhys,VoterPhysRev} and computational vibrational properties
beyond harmonic approximation \cite{HuangIyengar,Makri,Cao,Bowman,Gerber,
Jung,Matsunaga,McCoy1,McCoy2,IyengarParker,Meyer}. For a system containing
$N$ atoms, a full potential energy surface is a $3N$ dimensional function. Say,
we approximate the
surface by using a finite number of grid points {$M_1,M_2,...,M_{3N}$} along dimension 1,
dimension 2,..., dimension 3N respectively. This way we will end up calculating
{$\displaystyle\prod_{i=1}^{3N} M_{i}$} number of electronic structure calculations
to compute the full surface. Computing high dimensional potentials is an active
area of research \cite{Varandas-PES, truhlar, Bowman-IRPC-PES, ee-mbe-water,
conte2015permutationally, mancini2014new, wang2016five, conte2015trajectory,
yu2017high, Bowman-polybasis-fitting, aph, saller2014basis, reilly2010determination,
collins2011ab, evenhuis2005interpolation, zhou2011ab, qwaimd, sumner2007quantum,
qwaimd-SLO-1, fragAIMD, fragAIMD-elbo, fragAIMD-CC, behler2015constructing,
rupp2015machine, botu2015adaptive, kohonen1988introduction, handley2010potential,
manzhos2015neural, natarajan2015representing, behler2007generalized,
behler2011neural}. But the challenge remains and is at the fore-font of modern
computational chemistry.

In this document, we discuss solutions that are applicable to both problems
discussed above. This document is organized as follows: in chapter 2, we provide
a brief overview of fragment based AIMD. We then discuss limitations that reside
at the heart of potential energy surface calculations and also dynamics in many
reactive systems. Solutions are then discussed which combine multiple fragmentation
protocols simultaneously into a single calculation. We are strongly influenced here
by the literature of non-adiabatic dynamics, as we highlight. Specifically,
multiple topologies in a fragmentation calculation are interpolated as diabatic
states and the resultant method is multi-configurational in nature, where
diabatic states are combined to obtain a smooth, interpolated, adaptive,
system-independent, multi-topology-based fragmentation scheme. In Chapter 3 we
compute potential surfaces using the proposed method. Conclusions and future
directions for this work are given in Chapter 4. Two appendices, Appendix A and
Appendix B are attached at the end of the report. Appendix A describes
critical mathematical details. For example, it is found in Chapter 2 that the
method of computing a multi-topology surface requires non-Hermitian matrices.
It is shown in Appendix A how the class of matrices used here are in fact
always have a real spectral span. Appendix B describes analysis of an
algorithm to reuse results of electronic structure calculations in
calculation of potential surfaces.
\newpage



