\chapter{The need for a non-Hermitian topology matrix $M$}

The motivation for purely imaginary off-diagonal elements, and hence a non-Hermitian
topology matrix $M$, arises from the following
set of arguments. Let us take a 2 by 2 case of interpolation matrix.
Say, $E^{PIE-ONIOM}_1$ and $E^{PIE-ONIOM}_2$ are energies corresponding to the two
topologies 1 and 2 respectively and c is the off-diagonal element such that
$c = a \pm ib$.

\begin{enumerate}
\item Case I : Interpolation matrix is hermitian
\begin{equation}\label{eq2}
\left.\begin{aligned}
|M-\lambda I| &     =  u(\lambda)\\
              &     =  (E^{PIE-ONIOM}_1-\lambda)(E^{PIE-ONIOM}_2-\lambda) - (a-ib)(a+ib)\\
              &     =  \lambda^2 -(E^{PIE-ONIOM}_1 + E^{PIE-ONIOM}_2)\lambda + E^{PIE-ONIOM}_1E^{PIE-ONIOM}_2\\
              &~~~  - (a^2+b^2)
\end{aligned}\right.
\end{equation}

The above given function $u(\lambda)$ is a quadratic function which is a downward shifted
case of a parabolic function :
\begin{equation}\label{eq3}
\left.\begin{aligned}
v(\lambda) &     =  \lambda^2 -(E^{PIE-ONIOM}_1 + E^{PIE-ONIOM}_2)\lambda + E^{PIE-ONIOM}_1E^{PIE-ONIOM}_2
\end{aligned}\right.
\end{equation}

with positive curvature,
whose zeroes are $E^{PIE-ONIOM}_1$ and $E^{PIE-ONIOM}_2$. As the discriminant of the function
$u(\lambda)$ is always positive, the zeros will always exist. As $v(\lambda)$ parabola is
shifted downwards by $a^2+b^2$, the new zeros will lie outside the interval
[$E^{PIE-ONIOM}_1$,$E^{PIE-ONIOM}_2$]. Hence, the zero of the $u(\lambda)$ or the eigenvalues of $M$
cannot lie in [$E^{PIE-ONIOM}_1$,$E^{PIE-ONIOM}_2$].

\item Case II : Interpolation matrix is not hermitian
\begin{equation}\label{eq4}
\left.\begin{aligned}
|M-\lambda I| &     =  (E^{PIE-ONIOM}_1-\lambda)(E^{PIE-ONIOM}_2-\lambda) - (a+ib)(a+ib)\\
              &     =  \lambda^2 -(E^{PIE-ONIOM}_1 + E^{PIE-ONIOM}_2)\lambda + E^{PIE-ONIOM}_1E^{PIE-ONIOM}_2\\
              &     ~~~- (a^2-b^2+2iab)
\end{aligned}\right.
\end{equation}

For above mentioned function to have zeros on real axis its imaginary part has to be zero hence
either a or b should be equal to zero. If $b = 0$ then it is the same as case I, as the function
is a case of downward shifting of $f(\lambda)$, the roots cannot lie in between $E^{PIE-ONIOM}_1$
and $E^{PIE-ONIOM}_2$. Hence we will ignore this case. If $a = 0$ i.e. c is purely imaginary, the
condition of positivity on discriminant of the above function enforces the condition that
$|E^{PIE-ONIOM}_1-E^{PIE-ONIOM}_2| < |c|$. As the parabola is shifted upwards by $c^2$,
the new zeros(only if they exist) will lie inside the interval [$E^{PIE-ONIOM}_1$,$E^{PIE-ONIOM}_2$].
Hence, the zeros of the function $|M - \lambda I|$ or the eigenvalues of $M$ can lie between the
two diabatic energy curves.
\end{enumerate}

\newpage
